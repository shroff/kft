\documentclass[12pt]{article}
\begin{document}

\title{CS 3251 -- Programming Assignment 2}
\author{Abhishek Shroff}
\date{September 23, 2010}
\maketitle

\section{Submission}
This tar.gz file contains 4 directories -- \emph{bin}, \emph{doc}, \emph{src}, and \emph{test}.

The \emph{doc} directory contains this readme, as well as the output file that shows the replies received from the TCP and UDP servers.

The \emph{bin} directory contains binary files created as a result of executing 'make' in the \emph{src} directory.

The \emph{src} directory contains all the source files that are required for this project including a Makefile that contains a target to build the binaries and place them in the appropriate folder.

The \emph{test} directory contains test files of various sizes ranging from 100kb - 100mb, and some test scripts that record values after running the required transfer tests.

\section{Protocol}
The protocol follows an extremely simple design with minimal header information.

First, a transaction is started by a client by sending a packet whose first byte is a 0, followed by 3 bytes of maximum packet length, least significant byte first. Following these 4 bytes are the characters of the file name up until the end of the packet. Note that since the maximum packet size has to be sent in 3 bytes, a packet size cannot have an unsigned representation greater than 24 bits. This means that the maximum supported packet size is 4MB.

The server receives this packet and
\section{Limitations}
The servers are fairly robust, and will not terminate on any input from the clients. That said, there are some limitations, which, if surpassed, will break this program.
\begin{enumerate}
\item The size of data inside a single request may not exceed 511 bytes.
\item All operations are performed on a closed set of integers, which means floating points are not supported by either input or as a result of a calculation. Floating points will be truncated.
\end{enumerate}
\end{document}
